\section{Introduction}
Software development is a continuous process: developers incrementally add, change or remove code and store software revisions in a Version Control System (VCS). Each changeset (or: \textit{diff}) is provided with a commit message, which is a short, human-readable summary of the \textit{what} and \textit{why} of the change \citep{buse_automatically_2010}.

Commit messages document the development process and aid developers in their understanding of the state of the software and its evolution over time. However, developers do not always properly document code changes or write commit messages of low quality \citep{dyer_boa:_2013}. This applies to code documentation in general and negatively impacts developer performance \citep{roehm_how_2012}. Automatically generating high-quality commit messages from diffs would relieve the burden of writing commit messages off developers, and improve the documentation of the codebase.

As demonstrated by \citet{buse_automatically_2010}, \citet{cortes-coy_automatically_2014} and \citet{shen_automatic_2016}, predefined rules and text templates can be used to generate commit messages that are largely preferred to developer-written messages in terms of descriptiveness. However, the main drawbacks of these methods are that 1) only the \textit{what} of a change is described, 2) the generated messages are too comprehensive to replace short commit messages, and 3) these methods do neither scale nor generalize on unseen constructions, because of the use of hand-crafted templates and rules.

According to a recent study by \citet{jiang_towards_2017}, code changes and commit messages exhibit distinct patterns that can be exploited by machine learning. The hypothesis is that methods based on machine learning, given enough training data, are able to extract more contextual information and latent factors about the \textit{why} of a change. Furthermore, \citet{allamanis_survey_2018} state that source code is ``a form of human communication [and] has similar statistical properties to natural language corpora''. Following the success of (deep) machine learning in the field of natural language processing, neural networks seem promising for automated commit message generation as well.

\citet{jiang_automatically_2017} have demonstrated that generating commit messages with neural networks is feasible. This work aims to reproduce the results from \cite{jiang_automatically_2017} on the same and a different dataset. Additionally, efforts are made to improve upon these results by applying a different data processing technique. More specific, the following research questions will be answered:

\begin{itemize}
    \item[RQ1:] \textbf{Can the results from \citet{jiang_automatically_2017} be reproduced?}
    \item[RQ2:] \textbf{Does a more rigorous dataset processing technique improve the results of the neural model?}
\end{itemize}

This paper is structured as follows. In \Cref{sec:background}, background information about deep neural networks and neural machine translation is covered. In \Cref{sec:relatedwork}, the state-of-the-art in commit message generation from source code changes is reviewed. \Cref{sec:methodology} describes the implementation of the neural model. \Cref{sec:preprocessing} covers preprocessing techniques and analyzes the resulting dataset characteristics. \Cref{sec:results} presents the evaluation results and \Cref{sec:discussion} discusses the performance and limitations. \Cref{sec:conclusion} summarizes the findings and points to promising future research directions.
